% This document is based on a template created by Ted Pavlic (http://www.tedpavlic.com)


%----------------------------------------------------------------------------------------
%	PACKAGES AND OTHER DOCUMENT CONFIGURATIONS
%----------------------------------------------------------------------------------------

\documentclass{article}

\usepackage{fancyhdr} % Required for custom headers
\usepackage{lastpage} % Required to determine the last page for the footer
\usepackage{extramarks} % Required for headers and footers
\usepackage[usenames,dvipsnames]{color} % Required for custom colors
\usepackage{graphicx} % Required to insert images
\usepackage{subcaption}
\usepackage{listings} % Required for insertion of code
\usepackage{courier} % Required for the courier font
%\usepackage{lipsum} % Used for inserting dummy 'Lorem ipsum' text into the template
\usepackage{amsmath,siunitx,physics,amssymb}
\usepackage{placeins}
\usepackage{enumitem}
\usepackage{hyperref,cleveref}

% Margins
\topmargin=-0.45in
\evensidemargin=0in
\oddsidemargin=0in
\textwidth=6.5in
\textheight=9.0in
\headsep=0.25in

\linespread{1.1} % Line spacing

% Set up the header and footer
\pagestyle{fancy}
\lhead{\hmwkAuthorName} % Top left header
\chead{\hmwkClass\ : \hmwkTitle} % Top center head
%\rhead{\firstxmark} % Top right header
\lfoot{\lastxmark} % Bottom left footer
\cfoot{} % Bottom center footer
\rfoot{Page\ \thepage\ of\ \protect\pageref{LastPage}} % Bottom right footer
\renewcommand\headrulewidth{0.4pt} % Size of the header rule
\renewcommand\footrulewidth{0.4pt} % Size of the footer rule

%\setlength\parindent{0pt} % Removes all indentation from paragraphs

%----------------------------------------------------------------------------------------
%	DOCUMENT STRUCTURE COMMANDS
%	Skip this unless you know what you're doing
%----------------------------------------------------------------------------------------

% Header and footer for when a page split occurs within a problem environment
\newcommand{\enterproblemHeader}[1]{
%\nobreak\extramarks{#1}{#1 continued on next page\ldots}\nobreak
%\nobreak\extramarks{#1 (continued)}{#1 continued on next page\ldots}\nobreak
}

% Header and footer for when a page split occurs between problem environments
\newcommand{\exitproblemHeader}[1]{
%\nobreak\extramarks{#1 (continued)}{#1 continued on next page\ldots}\nobreak
%\nobreak\extramarks{#1}{}\nobreak
}

\setcounter{secnumdepth}{0} % Removes default section numbers
\newcounter{problem} % Creates a counter to keep track of the number of problems
\setcounter{problem}{-1}

\newcommand{\problemName}{}
\newenvironment{problem}[1][Part \theproblem]{ % Makes a new environment called problem which takes 1 argument (custom name) but the default is "problem #"
	\stepcounter{problem} % Increase counter for number of problems
	\renewcommand{\problemName}{#1} % Assign \problemName the name of the problem
	\section{\problemName} % Make a section in the document with the custom problem count
	\enterproblemHeader{\problemName} % Header and footer within the environment
}{
	\exitproblemHeader{\problemName} % Header and footer after the environment
}

\newcommand{\problemAnswer}[1]{ % Defines the problem answer command with the content as the only argument
	\noindent\framebox[\columnwidth][c]{\begin{minipage}{0.98\columnwidth}#1\end{minipage}} % Makes the box around the problem answer and puts the content inside
}

\newcounter{subproblem}[problem]
\newcommand{\subproblemName}{}
\newenvironment{subproblem}[1][\theproblem~(\alph{subproblem})]{ % New environment for sections within  problems, takes 1 argument - the name of the section
	\stepcounter{subproblem}
	\renewcommand{\subproblemName}{#1} % Assign \problemName the name of the problem
	\subsection{\subproblemName} % Make a section in the document with the custom problem count
	\enterproblemHeader{\subproblemName} % Header and footer within the environment
}{
	\enterproblemHeader{\problemName} % Header and footer after the environment
}

\newcommand{\numberthis}{\addtocounter{equation}{1}\tag{\theequation}}

%----------------------------------------------------------------------------------------
%	NAME AND CLASS SECTION
%----------------------------------------------------------------------------------------

\newcommand{\hmwkTitle}{Assignment\ \#$4$} % Assignment title
\newcommand{\hmwkDueDate}{Moday,\ April\ 2,\ 2018} % Due date
\newcommand{\hmwkClass}{CSC411} % Course/class
\newcommand{\hmwkClassTime}{} % Class/lecture time
\newcommand{\hmwkAuthorName}{Izaak Niksan and Lukas Zhornyak} % Your name

%----------------------------------------------------------------------------------------
%	TITLE PAGE
%----------------------------------------------------------------------------------------

\title{
	\vspace{2in}
	\textmd{\textbf{\hmwkClass:\ \hmwkTitle}}\\
	\normalsize\vspace{0.1in}\small{Due\ on\ \hmwkDueDate}\\
	\vspace{0.1in}
	\vspace{3in}
}

\author{\textbf{\hmwkAuthorName}}
%\date{} % Insert date here if you want it to appear below your name

%----------------------------------------------------------------------------------------

\begin{document}

\maketitle
\clearpage

%----------------------------------------------------------------------------------------
%	ENVIRONMENT
%----------------------------------------------------------------------------------------

\begin{problem}[Environment]	
	This project was created with Python 3.6.4 with numpy 1.14.1, scipy 1.0.0, scikit-image 0.13.1, and matplotlib 2.1.2, pytorch 0.3.0, torchvision 0.2.0, as well as all associated dependencies.
\end{problem}
\clearpage

%----------------------------------------------------------------------------------------
%	PART 1
%----------------------------------------------------------------------------------------
\FloatBarrier
\begin{problem}
By inspecting the code it was determined that the grid is represented by a numpy array of 9 elements, one for each space on the board. The parameter \textit{turn} represents whose turn it is, taking on values of 1 or 2 (for x and o respectively). The parameter \textit{done} is a Boolean value denoting whether or not the game is finished.\\\\
A game of Tic-Tac-Toe (albeit an amateur one) was simulated using the code seen in Listing 1. Its text output is seen below it.
\begin{lstlisting}[language=Python, caption=Simulated game of Tic-Tac-Toe]
p1_env=Environment()
print('First turns:')
p1_env.step(0)
p1_env.step(1)
p1_env.render()
print('Second turns:')
p1_env.step(8)
p1_env.step(5)
p1_env.render()
print('Third turn (x wins):')
p1_env.step(4)
p1_env.render()

...

First turns:
xo.
...
...
====
Second turns:
xo.
..o
..x
====
Third turn (x wins):
xo.
.xo
..x
====
\end{lstlisting}
	
	
\end{problem}
\clearpage

%----------------------------------------------------------------------------------------
%	PART 2
%----------------------------------------------------------------------------------------
\FloatBarrier
\begin{problem}
	
	\begin{subproblem}
		
	\end{subproblem}
	
	\begin{subproblem}
By inspecting the 27-dimensional vector


\\\\\\\\\\\\\\\\\\\

!!!!!!!!!!!!!!!!!!!!!!!!!!!!!!!!!!!!!!!!!!!!
	\end{subproblem}
	
	\begin{subproblem}
The output of this policy is a 9-dimensional vector, in the same format as \textit{grid} described in Part 1, with each value representing the probability of that spot being chosen this turn. In other words, it represents a discrete probability distribution. Since \textit{select\_action} takes a sample from this distribution, and does not always choose the element of highest probability, the policy is therefore stochastic.
	\end{subproblem}
	
\end{problem}
\clearpage

%----------------------------------------------------------------------------------------
%	PART 3
%----------------------------------------------------------------------------------------
\FloatBarrier
\begin{problem}
	
	\begin{subproblem}
		
	\end{subproblem}
	
	\begin{subproblem}
		
	\end{subproblem}
	
\end{problem}
\clearpage

%----------------------------------------------------------------------------------------
%	PART 4
%----------------------------------------------------------------------------------------
\FloatBarrier
\begin{problem}
	
	\begin{subproblem}
		
	\end{subproblem}
	
	\begin{subproblem}
		
	\end{subproblem}
	
\end{problem}
\clearpage

%----------------------------------------------------------------------------------------
%	PART 5
%----------------------------------------------------------------------------------------
\FloatBarrier
\begin{problem}
	
	\begin{subproblem}
		
	\end{subproblem}
	
	\begin{subproblem}
		
	\end{subproblem}
	
	\begin{subproblem}
		
	\end{subproblem}
	
	\begin{subproblem}
		
	\end{subproblem}
	
\end{problem}
\clearpage

%----------------------------------------------------------------------------------------
%	PART 6
%----------------------------------------------------------------------------------------
\FloatBarrier
\begin{problem}
	
	
\end{problem}
\clearpage

%----------------------------------------------------------------------------------------
%	PART 7
%----------------------------------------------------------------------------------------
\FloatBarrier
\begin{problem}
	
	
\end{problem}
\clearpage

%----------------------------------------------------------------------------------------
%	PART 8
%----------------------------------------------------------------------------------------
\FloatBarrier
\begin{problem}
	
	
\end{problem}
\clearpage


%----------------------------------------------------------------------------------------

\end{document}